\documentclass{article}
\usepackage[utf8]{inputenc}
\usepackage{braket}
\usepackage{amsmath,amssymb}
\begin{document}

\subsection*{Braket notation}
State vector (Ket): 
$$\Ket{a} = 
\begin{bmatrix}
    a_0\\
    a_1\\
    \vdots\\
    a_n
\end{bmatrix}
$$
Conjugate transpose of state vector (Bra): 
$$\Bra{a} = 
\begin{bmatrix}
    a^*_0 & a^*_1 & \dots & a^*_n
\end{bmatrix}
$$
Addition:
$$\Ket{a}+\Ket{b}=
\begin{bmatrix}
    a_0+b_0\\
    a_1+b_1\\
    \vdots\\
    a_n+b_n
\end{bmatrix}
$$
Scalar Multiplication:
$$
    x\Ket{a}=
    \begin{bmatrix}
        x\times a_0\\
        x\times a_1\\
        \vdots\\
        x\times a_n
    \end{bmatrix}
$$
Inner Product:
$$\Braket{a|b}=a^*_0b_0+a^*_1b_1+\dots+a^*_nb_n$$
Outer Product:
$$\Ket{a}\Bra{b}=
\begin{bmatrix}
    a_0\\
    a_1\\
    \vdots\\
    a_n
\end{bmatrix}
\begin{bmatrix}
    b^*_0 & b^*_1 & \dots & b^*_n
\end{bmatrix}
=
\begin{bmatrix}
    a_0b^*_0 & a_0b^*_1 & \dots  & a_0b^*_n\\
    a_1b^*_0 & a_1b^*_1 &        & \vdots\\
    \vdots   &          & \ddots & \vdots\\
    a_nb^*_0 & a_nb^*_1 & \dots  & a_nb^*_n
\end{bmatrix}
$$
Tensor Product:
$$
\Ket{ab}=\Ket{a}\otimes\Ket{b}=
\begin{bmatrix}
    a_0 & \begin{bmatrix} b_0\\b_1\end{bmatrix}\\
    a_1 & \begin{bmatrix} b_0\\b_1\end{bmatrix}
\end{bmatrix}=
\begin{bmatrix}
    a_0b_0\\
    a_0b_1\\
    a_1b_0\\
    a_1b_1\\
\end{bmatrix}
$$
\newpage
\subsection*{Qubit State Vector}
$$\Ket{q} = \alpha\Ket{0} + \beta\Ket{1} =
\begin{bmatrix}
    \alpha\\
    \beta
\end{bmatrix}
$$
$ \alpha \in \mathbb{C}$ being the probability it collapse to 0 when measured\\
$ \beta  \in \mathbb{C}$ being the probability it collapse to 1 when measured\\

The state that will always measured to be 0:
$$
\Ket{0} = 
\begin{bmatrix}
    1\\
    0
\end{bmatrix}
$$

The state that will always measured to be 1:
$$
\Ket{1} = 
\begin{bmatrix}
    0\\
    1
\end{bmatrix}
$$

The probability of measuring a state $\Ket{\psi}$ in the state $\Ket{x}$:
$$p(\Ket{x})=|\braket{x|\psi}|^2$$

Qubit represented in $\theta,\phi\in\mathbb{R}$:
$$\Ket{q}=\cos\frac{\theta}{2}\Ket{0}+e^{i\phi}\sin\frac{\theta}{2}\Ket{1}$$

\newpage
\subsection*{Single Quantum Gate}
\subsubsection*{Pauli Gates}
X-Gate: Rotate by $\pi$ around the x-axis of Bloch sphere
$$
X=
\begin{bmatrix}
    0 & 1\\
    1 & 0
\end{bmatrix}
=\Ket{1}\Bra{0}+\Ket{0}\Bra{1}
$$
Equivalent to classical NOT when operated on classical states:
$$X\Ket{0}=\left(\Ket{0}\Bra{1}+\Ket{1}\Bra{0}\right)\Ket{0}=\Ket{0}\Braket{1|0}+\Ket{1}\Braket{0|0}=\Ket{0}\times0+\Ket{1}\times1=\Ket{1}$$
$$X\Ket{1}=(\Ket{0}\Bra{1}+\Ket{1}\Bra{0})\Ket{1}=\Ket{0}\Braket{1|1}+\Ket{1}\Braket{0|1}=\Ket{0}\times1+\Ket{1}\times0=\Ket{0}$$
X-basis (eigenstates of X-gate):
$$
\Ket{+}=\frac{1}{\sqrt{2}}(\Ket{0}+\Ket{1})
=\frac{1}{\sqrt{2}}
\begin{bmatrix}
    1\\
    1
\end{bmatrix}
$$
$$
\Ket{-}=\frac{1}{\sqrt{2}}(\Ket{0}-\Ket{1})
=\frac{1}{\sqrt{2}}
\begin{bmatrix}
    1\\
    -1
\end{bmatrix}
$$
syntax (qc is of object QuantumCircuit, qubit is index of qubit in the circuit to operate on): 
\begin{verbatim*}
qc.x(qubit)    
\end{verbatim*}
Y-Gate: Rotate by $\pi$ around the Y-axis of Bloch sphere
$$
Y=
\begin{bmatrix}
    0 & -i\\
    i & 0
\end{bmatrix}
=i\Ket{1}\Bra{0}-i\Ket{0}\Bra{1}
$$
Y-basis:
$$
\Ket{\circlearrowleft}=\frac{1}{\sqrt{2}}(\Ket{0}+i\Ket{1})
=\frac{1}{\sqrt{2}}
\begin{bmatrix}
    1\\
    i
\end{bmatrix}
$$
$$
\Ket\circlearrowright=\frac{1}{\sqrt{2}}(\Ket{0}-i\Ket{1})
=\frac{1}{\sqrt{2}}
\begin{bmatrix}
    1\\
    -i
\end{bmatrix}
$$
syntax: 
\begin{verbatim*}
qc.y(qubit)    
\end{verbatim*}
Z-Gate: Rotate by $\pi$ around the Z-axis of Bloch sphere
$$
Z=
\begin{bmatrix}
    1 & 0\\
    0 & -1
\end{bmatrix}
=\Ket{0}\Bra{0}-\Ket{1}\Bra{1}
$$
Z-basis: 
$$
\Ket{0}=
\begin{bmatrix}
    1\\
    0
\end{bmatrix}
$$$$
\Ket{1}=
\begin{bmatrix}
    0\\
    1
\end{bmatrix}
$$
syntax: 
\begin{verbatim*}
qc.z(qubit)    
\end{verbatim*}
\newpage
\subsubsection*{Hadamard Gate (H-gate)}
Rotation around the Bloch vector [1,0,1] (line between x and z-axis)
$$H=\frac{1}{\sqrt{2}}
\begin{bmatrix}
    1 & 1\\
    1 & -1
\end{bmatrix}=\Ket{+}\Bra{0}+\Ket{-}\Bra{1}$$
Transform Z-basis ($\Ket{0}$ and $\Ket{1}$) to X-basis ($\Ket{+}$ and $\Ket{-}$):
$$H\Ket{0}=\left(\Ket{+}\Bra{0}+\Ket{-}\Bra{1}\right)\Ket{0}=\Ket{+}\Braket{0|0}+\Ket{-}\Braket{1|0}=\Ket{+}\times1+\Ket{-}\times0=\Ket{+}$$
$$H\Ket{1}=\left(\Ket{+}\Bra{0}+\Ket{-}\Bra{1}\right)\Ket{1}=\Ket{+}\Braket{0|1}+\Ket{-}\Braket{1|1}=\Ket{+}\times0+\Ket{-}\times1=\Ket{-}$$
syntax: 
\begin{verbatim*}
qc.h(qubit)    
\end{verbatim*}
\subsubsection*{$R_\phi$-gate}
Rotation of $\phi$ around z-axis
$$R_\phi=
\begin{bmatrix}
    1 & 0\\
    0 & e^{i\phi}
\end{bmatrix}=\Ket{0}\Bra{0}+e^{i\phi}\Ket{1}\Bra{1}
$$
syntax (phi is $\phi$ in radian): 
\begin{verbatim*}
qc.rz(phi, qubit)    
\end{verbatim*}
Identity Gate (I-gate)
$$I=
\begin{bmatrix}
    1 & 0\\
    0 & 1
\end{bmatrix}=\Ket{0}\Bra{0}+\Ket{1}\Bra{1}$$
syntax: 
\begin{verbatim*}
qc.i(qubit)    
\end{verbatim*}

\end{document}